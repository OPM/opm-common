\newcommand{\pyaction}{\kw{PYACTION}}
\chapter{Programming in the deck: \pyaction{}}
\label{pyaction}
The \pyaction{} keyword is a \flow{} specific keyword which allows for Python
programming in the \kw{SCHEDULE} section. The \pyaction{} keyword is inspired by
the \actionx{} keyword, but instead of a \inlinecode{.DATA} formatted condition
you are allowed to implement the condition with a general Python script. The
\actionx{} keywords are very clearly separated in a condition part and an action
part in the form of a list of keywords which are effectively injected in the
\kw{SCHEDULE} section when the condition evaluates to true. This is not so for
\pyaction{} where there is only one Python script which can both evaluate
conditions and apply changes. In principle the script can run arbitrary code,
but due to the complexity of the \kw{SCHEDULE} datamodel the ``current best''
way to actually change the course of the simulation is through the use of an
additional dummy \actionx{} keyword.

In order to enable the \pyaction{} keyword \flow{} must be compiled with the
\path{cmake} switches \inlinecode{-DOPM\_ENABLE\_EMBEDDED\_PYTHON=ON} and
\inlinecode{-DOPM\_ENABLE\_PYTHON=ON}, the default is to build with these switches
set to \inlinecode{OFF}. Before you enable \pyaction{} in your \flow{}
installation please read carefully through section \ref{pyaction_security} for
security implications of \pyaction{}.

\section{Python - wrapping and embedding}
Python is present in the \flow{} codebase in two different ways. For many of
the classes in the \flow{} codebase - in particular in opm-common, there are
\emph{Python wrappers} available. That means that you can invoke the C++
functionality in \flow{} classes from Python - e.g. this Python script can be used to
load a deck and print all the keywords:

\begin{code}
import sys
from opm.io.parser import Parser

input_file = sys.argv[1]
parser = Parser()

deck = parser.parse_file(input_file)
for kw in deck:
    print(kw.name)
\end{code}

When used this way the Python interpreter is the main program running, and the
\flow{} classes like \inlinecode{Opm::Parser} are loaded to extend the Python
interpreter. This can also be flipped around, the Python interpreter can be
\emph{embedded} in the \flow{} executable. When Python is embedded, \flow{} is
the main program running, and with help of the embedded interpreter the \flow{}
program can be extended with Python plugins. The \pyaction{} keyword can be
perceived as a Python plugin. To really interact with the state of the \flow{}
simulation the plugin needs to utilize the functionality which wraps the C++
functionality, so for \pyaction{} both wrapping and embedding is at play.

Exporting more functionality from C++ to Python in the form of new and updated
wrappers is a quite simple and mechanical process. If you need a particular
functionality which is already available in C++ also in Python it will probably
be a quite limited effort for a developer who is already familiar with the code.


\section{The \pyaction{} keyword}
The \pyaction{} keyword is in the \kw{SCHEDULE} section like \actionx{}. The
first record is the name of the action and a string identifier for how many
times the action should run, then there is a path to a Python module:

\begin{deck}
PYACTION
  PYTEST 'FIRST_TRUE' /
  'pytest.py' /
\end{deck}

This keyword defines a \pyaction{} called \kw{PYTEST} which will run at the end
of every timestep until the first time a \inlinecode{true} value is returned. In
addition to \kw{FIRST\_TRUE} you can choose \kw{SINGLE} to run exactly once and
\kw{UNLIMITED} to continue running at the end of every timestep for the entire
simulation. The second record is the path to a file with Python code which will
run when this \pyaction{} is invoked. The path to the module will be interpreted
relative to the location of the \path{.DATA} file.

The python module can be quite arbitrary, but it must contain a function
\inlinecode{run} with the correct signature:
\begin{code}
def run(ecl_state, schedule, report_step, summary_state, actionx_callback):
    print('Running python code in PYACTION')
    return True
\end{code}
The \pyaction{} machinery is not as robust as the simulator proper: while
loading the \kw{PYACTION} keyword \flow{} will check that the Python module
contains syntactically valid Python code, and that it contains a
\inlinecode{run()} function, but it will \emph{not} check the signature of the
\inlinecode{run()} function. If the signature is wrong you will get a hard to
diagnose runtime error.

When the Python module is loaded it does so in an environment where the path to
the \path{.DATA} file has been appended to the Python load path by manipulating
the internal \inlinecode{sys.path} variable.


\subsection{The different arguments}
The \inlinecode{run()} function will be called with exactly five arguments which
your implementation can use. These arguments point to datastructures in the
simulator, and is the way to interact with the state of the simulation. The five
arguments are:
\begin{description}
\item[\inlinecode{ecl\_state}:] An instance of the \inlinecode{Opm::EclipseState}
  class - this is a representation of \emph{all static properties} in the model,
  ranging from porosity to relperm tables. The content of the
  \inlinecode{ecl\_state} is immutable - you are not allowed to change the static
  properties at runtime\footnote{This could certainly be interesting, but this
  is beyond the scope of the \pyaction{} keyword.}.
\item[\inlinecode{schedule}:] An instance of the \inlinecode{Opm::Schedule}
  class - this is a representation of all the content from the \kw{SCHEDULE}
  section, notably all well and group information and the timestepping. Being
  able to change the \kw{SCHEDULE} information runtime is certainly one of the
  main motivations for this functionality, however due to the complexity of
  the \inlinecode{Opm::Schedule} class (section \ref{schedule_design})
  the recommended way to actually mutate the \inlinecode{Opm::Schedule} is
  through the use of a dummy \actionx{} keyword (section
  \ref{pyaction_actionx}).
\item[\inlinecode{report\_step}:] This is an integer for the report step we
  are currently working on. Observe that the \pyaction{} is called for every
  simulator timestep, i.e. it will typically be called multiple times with
  the same value for the $\mathrm{report\_step}$ argument.
\item[\inlinecode{summary\_state}:] An instance of the
  \inlinecode{Opm::SummaryState} class, this is where the current summary
  results of the simulator are stored. The \inlinecode{SummaryState} class has
  methods to get hold of well, group and general variables
  \begin{code}
    # Print all well names
    for well in summary_state.wells:
        print(well)

    # Assign all group names to the variable group_names
    group_names = summary_state.groups

    # Sum the oil rate from all wells.
    sum_wopr = 0
    for well in summary_state.wells:
        sum_wopr += summary_state.well_var(well, 'WOPR')

    # Directly fetch the FOPR from the summary_state
    fopr = summary_state['FOPR']
  \end{code}
  The \inlinecode{summary\_state} variable can also be updated with the
  \inlinecode{update()}, \inlinecode{update\_well\_var()} and
  \inlinecode{update\_group\_var()} methods.
\item[\inlinecode{actionx\_callback}:] The \inlinecode{actionx\_callback} is a
  specialized function which is used to update the \inlinecode{Schedule} object
  by applying the keywords from a normal \actionx{} keyword. This is described
  in detail in section \ref{pyaction_actionx}.
\end{description}

\subsection{Holding state}
The \pyaction{} keywords will often be invoked multiple times, a Python
dictionary \inlinecode{state} has been injected in the module - that dictionary
can be used to maintain state between invocations. Let us assume we want to
detect when the field oil production starts curving down - i.e. when
$\partial^2_{t} \mathrm{FOPR} < 0$, in order to calculate that we need to keep
track of the timesteps and the $\mathrm{FOPR}$ as function of time - this is one
possible implementation:
\begin{code}
def diff(pair1, pair2):
    return (pair1[0] - pair2[0], pair1[1] - pair2[1])

def fopr_diff2(summary_state):
    fopr = summary_state.get('FOPR')
    sim_time = summary_state.get('TIME')
    if not 'fopr' in state:
        state['fopr'] = []
    fopr_series = state['fopr']
    fopr_series.append( (sim_time, fopr) )

    if len(fopr_series) < 2:
       return None

    pair0 = fopr_series[-1]
    pair1 = fopr_series[-2]
    pair2 = fopr_series[-3]

    dt1, df1 = diff(pair0, pair1)
    dt2, df2 = diff(pair1, pair2)

    return 2*(df1/dt1 - df2/dt2)/(dt1 + dt2)

def run(ecl_state, schedule, report_step, summary_state, actionx_callback):
    fopr_d2 = fopr_diff2(summary_state)
    if not fopr_d2 is None:
       if fopr_d2 < 0:
           print('Hmmm - this is going the wrong way')
       else:
           print('All good - sky is the limit!')
\end{code}

\section{Changing the \inlinecode{Schedule} object - using a ``normal'' \actionx{}}
\label{pyaction_actionx}
Before reading this section you should make sure to understand the
\inlinecode{Schedule} design described in section \ref{schedule_design}. The
initial plan when implementing the \pyaction{} keyword was to be able to make
function calls like
\begin{code}
  schedule.close_well(w1, report_step)
  schedule.set_orat(w2, 1000, report_step)
\end{code}
to close a well and set the oil rate of another well. Unfortunately it proved
very complex to get good semantics for combining such runtime changes with the
keyword based model for \kw{SCHEDULE} section. The current recommendation is to
apply changes to the \kw{SCHEDULE} section using callbacks to \kw{ACTIONX}
keywords from Python code, this is illustrated in the example
below\footnote{From a programmers point of view the solution seems very
unsatisfactory, but it works and it plays nicely with the \kw{ACTIONX} behavior.
If/when the underlying \inlinecode{Schedule} implementation changes there is
nothing per se in the \pyaction{} design which inhibits use of a better
\inlinecode{Schedule} api in the future.}.

The recommended way to achieve this is to create a normal \actionx{} keyword
which is set up to run zero times, and then explicitly invoke that from the
Python \inlinecode{run()} function. In the example below we create an \actionx{}
\inlinecode{CLOSEWELLS} which will close all matching wells (the wellname '?')
\begin{deck}
ACTIONX
  CLOSEWELLS 0 /
  /
/

WELOPEN
  '?' 'CLOSE' /
/

ENDACTIO
\end{deck}
The \inlinecode{CLOSEWELLS} action is set up to run zero times, so the normal
\actionx{} machinery will never run this action\footnote{The \kw{CLOSEWELL}
action has an \emph{empty condition}, the \actionx{} keywords with empty
condition will always evaluate as false.}. Then in the Python run function we go
through all the wells and call the \inlinecode{CLOSEWELL} action to close those
with \inlinecode{OPR < 1000}:
\begin{code}
def run(ecl_state, schedule, report_step, summary_state, actionx_callback):
    close_wells = []
    for well in summary_state.wells:
        if summary_state.well_var(well, 'WOPR') < 1000:
           close_wells.append(well)

    if close_wells:
        actionx_callback('CLOSEWELLS', close_wells)
\end{code}
The implementation of this is quite complex with thread of execution going from
C++ to Python, then invoking a callback to C++ which will call
\inlinecode{Schedule::iterateScheduleSection()}, going back to Python to
complete the \inlinecode{run()} method before the function pointers pops back to
C++ and continues the simulator execution\footnote{This is documented in some
detail as code comments of \inlinecode{Schedule::applyPyAction()} in the
\path{Schedule.cpp} file.}.


\section{Implementing \udq{} like behavior}
The \udq{} keyword has three different purposes - all based on defining
complex quantities from the current state of the simulation:

\begin{enumerate}
\item Define a complex quantity to be used in a \actionx{} condition.
\item Define a complex quantity for reporting in the summary file.
\item Define a quantity which can used as a control in \kw{UDA}.
\end{enumerate}

All of these can be achieved by using the \pyaction{} keyword, although for the
two latter alternatives you must specify the \udq{} keyword in the deck first,
but you can let the \pyaction{} implementation override the value:
\begin{deck}
-- Observe that this UDQ will be assigned from a PYACTION keyword,
-- the value used in the ASSIGN statement below is pure dummy.
UDQ
  ASSIGN WUGOOD 1 /
  ASSIGN FUGOOD 1 /
/
\end{deck}

\subsection{Using \pyaction{} instead of \udq{} + \actionx{}}
Towards the end of section \ref{actionx_structure} it is demonstrated how \udq{}
and \actionx{} can be combined to implement an action in case a complicated
condition applies. As described in section \ref{pyaction_actionx} the best way
to actually invoke changes on the \kw{SCHEDULE} section is through the use of a
dummy \actionx{} keyword, but \pyaction{} is very well suited to evaluate
complex conditions. In the example below we close all wells which have
consistently produced less than 1000 $\mathrm{m^3/day}$ for more than 60 days:

\begin{code}
wopr_limit = 1000
time_limit = 60 * 3600 * 24

def init_state(summary_state):
    if 'closed_wells' in state:
        return

    state['closed_wells'] = set()
    bad_wells = {}
    for well in summary_state.wells:
        bad_wells[well] = None
    state['bad_wells'] = bad_wells


def run(ecl_state, schedule, report_step, summary_state, actionx_callback):
    shut_wells = []
    init_state(summary_state)
    for well in summary_state.wells:
        if well in state['closed_wells']:
           continue

        if summary_state.well_var(well, 'WOPR') < wopr_limit:
            elapsed = summary_state.elapsed()
            if state['bad_wells'][well] is None:
                state['bad_wells'][well] = elapsed
            else:
                bad_time = elapsed - state['bad_wells'][well]
                if bad_time > time_limit:
                    shut_wells.append(well)
                    state['closed_wells'].add( well )
        else:
            state['bad_wells'][well] = None

    if shut_wells:
       actionx_callback(shut_wells)

\end{code}

\subsection{Using \pyaction{} to report to the summary file}
The important point when using \pyaction{} to report complex results to the
summary file is just that the \inlinecode{summary\_state} argument to the
\inlinecode{run()} function is \emph{writable} with \inlinecode{updata\_xxx}
calls. Assuming dummy \udq{} variables \kw{WUGOOD} and \kw{FUGOOD} have been
defined as per the example above, we can use \pyaction{} to set variable
\kw{FUGOOD} to one for all wells with rate above a limit, and the \kw{FUGOOD}
variable can be the count of such wells:
\begin{code}
def run(ecl_state, schedule, report_step, summary_state, actionx_callback):
    good_count = 0
    opr_limit = 1000
    for wname in schedule.well_names():
        if summary_state.well_var(wname, 'FOPR') > opr_limit:
            good_count += 1
            summary_state.update_well_var(wname, 'WUGOOD', 1)
        else:
            summary_state.update_well_var(wname, 'WUGOOD', 0)

    summary_state.update_var('FUGOOD', good_count)
\end{code}



\subsection{Using \pyaction{} to set a \kw{UDA} control}
Using \pyaction{} to set \kw{UDA} controls is quite simple. Again the \udq{}
keyword must have been defined with a dummy value in the \kw{SCHEDULE} section,
and the \kw{UDA} keyword used in e.g. a \kw{WCONDPROD} keyword. Then the
\inlinecode{run()} function can just be used to assign to the \udq{} variable.
In the example below we use a \kw{UDA} to control the oil production rate, and
the value is set to the average value of the producing wells:
\begin{deck}
-- Define dummy UDQ WUOPR to be used as control in the WCONPROD
-- keyword. The actual value for this UDQ is assigned in a PYACTION
-- keyword
UDQ
   ASSIGN WUOPR 0 /
/
...
...
-- Need to define a well list with all the production wells.
-- This is to ensure that the WCONPROD keyword is only applied
-- to producers.
WLIST
  'PROD'  P1 P2 P3 .../

WCONPROD
   '*PROD'  'OPEN'  'ORAT'  'WUOPR' /
/
\end{deck}
This can then be combined with the python code:
\begin{code}
def run(ecl_state, schedule, report_step, summary_state, actionx_callback):
    num_prod_wells = 0
    for wname in schedule.well_names():
        if summary_state.well_var(wname, 'WOPR') > 0:
            num_prod_wells += 1

    fopr = summary_state['FOPR']
    new_rate = fopr / num_prod_wells
    for wname in schedule.well_names():
        summary_state.update_well_var(wname, 'WUOPR', new_rate)
\end{code}


\section{Security implications of \pyaction{}}
\label{pyaction_security}
The \pyaction{} keyword allows for execution of arbitrary user supplied Python
code, with the priviliges of the user actually running \flow{}. If you have a
setup where \flow{} runs with a different user account than the person
submitting the simulation you should be \emph{very careful} about enabling the
embedded Python functionality and the \pyaction{} keyword. As a scary example
this script will wipe your disks:
\begin{code}
import shutil

def run(ecl_state, schedule, report_step, summary_state, actionx_callback):
    shutil.rmtree('/')
\end{code}

If the user running \flow{} has different security credentials than the user
submits the job, this has significant security implications.
